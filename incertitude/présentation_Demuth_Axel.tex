\documentclass[10pt]{beamer}
\usetheme{metropolis}
\usepackage[T1]{fontenc}
\usepackage[utf8]{inputenc}
\usepackage{graphicx}
\usepackage{stmaryrd}

\title{Présentation étude d'incertitudes sur l'équation de Burgers}
\author{Demuth Axel}
\date{\today}

\begin{document}

\maketitle

\section{Introduction}
\begin{frame}{Introduction}
    On étudie l'importance des paramètres influençant la dynamique d'un bouchon de circulation, modélisé par l'équation de Burgers suivante : 

    \begin{equation}
        \frac{\partial \rho}{\partial t} + \frac{\partial}{\partial x} \big(\rho v(\rho)\big) = 0,
        \end{equation}
        avec 
        \begin{equation}
        v(\rho) = v_{\text{max}} \left( 1 - \frac{\rho}{\rho_{\text{max}}} \right),
        \end{equation}
        et 
        \begin{equation}
        \rho(t = 0, x) =
        \begin{cases} 
        \rho_{\text{in}}, & x < x_C, \\
        \rho_{\text{out}}, & x > x_C.
        \end{cases}
    \end{equation}
        
\end{frame}


\end{document}
