\documentclass[a4paper,10pt]{article}
\usepackage[utf8]{inputenc}
\usepackage[T1]{fontenc}
\usepackage[francais]{babel}
\usepackage{lmodern}
\usepackage{geometry}
\geometry{a4paper, margin=1in}
\usepackage{enumitem}

\begin{document}

\title{\huge{Curriculum Vitae}}
\author{Demuth Axel}
\date{}
\maketitle

\noindent
\textbf{Nom :} Demuth Axel \\
\textbf{Adresse :} 11 Rue du Général de Gaulle, Gerstheim \\
\textbf{Téléphone :} 07 71 08 67 12 \\
\textbf{Email :} axeldemuth67150@gmail.com \\

\section*{Profil}
Étudiant en deuxième année de Master en Mathématiques Appliquées, à la recherche d'un stage de 4 à 6 mois entre février et août. Passionné par la simulation, la modélisation, et les mathématiques computationnelles, je possède de solides compétences en programmation et en analyse numérique.

\section*{Formation}
\noindent
\textbf{Master en Mathématiques Appliquées} \\
Université de Strasbourg, 2023 - en cours \\
Spécialisation en calcul scientifique et mathématiques de l'innovation, avec des compétences en simulation, modélisation et data science. \\
Principales matières : Analyse fonctionnelle, Optimisation, Équations aux dérivées partielles (EDP), Intelligence artificielle (IA), Calcul haute performance (HPC), Analyse de signal, Analyse de graphes. \\
Langages de programmation : Python, C++, Julia.

\noindent
\textbf{Licence en Mathématiques et Informatique} \\
Université de Strasbourg, 2019 - 2023 \\
Cours suivis : Algèbre, Calcul différentiel, Probabilités, Statistiques, Équations différentielles ordinaires (EDO), Analyse de Fourier, Programmation en Python et C++.

\section*{Langues}
\begin{itemize}[label={}]
    \item \textbf{Français :} Natif
    \item \textbf{Anglais :} Niveau B2
\end{itemize}

\section*{Expériences}
\noindent
\textbf{Stage - IRMA Strasbourg} \\
Juin à août 2024 \\
Développement d'une bibliothèque pour le projet Ktirio Urban Building dans le cadre du projet Numpex Exa-MA. Contribution au module Kinetic, en utilisant la bibliothèque CGAL et l'algorithme de reconstruction de surface cinétique pour reconstruire des maillages 3D à partir de nuages de points, avec correction des imprécisions du maillage.

\noindent
\textbf{Projet - Simulation de la chaleur sur un CPU} \\
Réalisation d'une simulation en C++ du comportement thermique d'un CPU à l'aide de solutions numériques d'EDP. Implémentation de deux modèles : stationnaire et dynamique.


\noindent
\textbf{Emplois étudiants}
\begin{itemize}
    \item Équipier polyvalent, McDonald's Erstein, été 2023
    \item Livreur de journaux, DNA, été 2022
\end{itemize}

\end{document}
