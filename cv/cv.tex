\documentclass[a4paper,10pt]{article}
\usepackage[utf8]{inputenc}
\usepackage{geometry}
\geometry{margin=1in}
\usepackage{lmodern}
\usepackage{enumitem}
\usepackage{titlesec}
\usepackage{titling}
\usepackage{parskip}

% Commande pour information personnelle
\newcommand{\personalinfo}[1]{
    \noindent
    \begin{tabbing}
    \hspace{2cm} \= \kill
    #1
    \end{tabbing}
}

% Commande pour ajouter une ligne sous chaque section
\newcommand{\sectionline}{
    \noindent\rule{\textwidth}{0.5pt}
}

\begin{document}

% Informations personnelles
\personalinfo{
    \textbf{Nom:} Demuth Axel \\
    \textbf{Adresse:} 11 Rue du général de Gaulle Gerstheim  \\
    \textbf{Téléphone:} 07 71 08 67 12
    \textbf{Email:} axeldemuth67150@gmail.com
}

% Profil
\section*{Profil}
\sectionline
Un paragraphe ici décrivant brièvement votre profil professionnel, vos compétences principales et vos objectifs de carrière.

% Formation
\section*{Formation}
\sectionline
\textbf{Etude:} Master Calcul scientifique et Mathématique de l'innovation (CSMI): en cours.\\
Description courte de vos études, spécialisations, projets pertinents, etc.

% Expérience
\section*{Expérience}
\sectionline
\textbf{Stage: } IRMA Strasbourg,juin à aout 2024  \\
Création d'une librairie pour le projet Ktirio Urban Building dans le cadre du projet Numpex Exa-MA.
Contribuant au module Kinetic,utilisant la bibliotèque CGAL et l'algorithme Kinetic surface reconstruction 
pour reconstruire des maillages 3D en partant d'un nuages de points capable de réparer les imprécisions du maillage.

\noindent
\textbf{Projet: } Simulation de Chaleur sur un CPU\\
Projet c++ visant à créer une simulation du comportement d'une partie d'un CPU en utilisant des solutions numérique EDP.En implémentant deux modèles : stationnaires et dynamique.


% Compétences
\section*{Compétences}
\sectionline
Listez ici vos compétences techniques, linguistiques, ou autres (ex: Langages de programmation, logiciels, langues, etc.).

\end{document}
