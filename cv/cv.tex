\documentclass[a4paper,10pt]{article}
\usepackage[utf8]{inputenc}
\usepackage{geometry}
\geometry{margin=1in}
\usepackage{lmodern}
\usepackage{enumitem}
\usepackage{titlesec}
\usepackage{titling}
\usepackage{parskip}

% Commande pour information personnelle
\newcommand{\personalinfo}[1]{
    \noindent
    \begin{tabbing}
    \hspace{2cm} \= \kill
    #1
    \end{tabbing}
}

% Commande pour ajouter une ligne sous chaque section
\newcommand{\sectionline}{
    \noindent\rule{\textwidth}{0.5pt}
}

\begin{document}

% Informations personnelles
\personalinfo{
    \textbf{Nom:} Demuth Axel \\
    \textbf{Adresse:} 11 Rue du général de Gaulle Gerstheim  \\
    \textbf{Téléphone:} 07 71 08 67 12 \\
    \textbf{Email:} axeldemuth67150@gmail.com
}

% Profil
\section*{Profil}
\sectionline
Etudiant en 2ème année de Master Mathématiques appliquées à la recherche d'un stage de 4 à 6 mois sur la période février-août.\\

% Formation
\section*{Formation}
\sectionline
\textbf{Master Mathématiques appliquées} 2023-en cours\\
Université de Strasbourg\\
Master de Calcul scientifique et Mathématiques de l'innovation, spécialisé en simulation,modelisation et data avec des compétences en informatiques.\\
Avec comme principale matière Analyse fonctionnelle,optimisation,EDP,IA,HPC,analyse de signal,de graphe,langage de programmation python,c++,julia.

\textbf{Licence de Mathématiques Informations} 2019-2023\\
Université de Strasbourg\\
Algebre,Calcul différentielle,probabilité,statistiques,EDO,Analyse de Fourrier,code en python,c++.

\textbf{Language :} 
Français (Natif), Anglais B2
% Expérience
\section*{Expérience}
\sectionline
\textbf{Stage: } IRMA Strasbourg,juin à aout 2024  \\
Création d'une librairie pour le projet Ktirio Urban Building dans le cadre du projet Numpex Exa-MA.
Contribuant au module Kinetic,utilisant la bibliotèque CGAL et l'algorithme Kinetic surface reconstruction 
pour reconstruire des maillages 3D en partant d'un nuages de points capable de réparer les imprécisions du maillage.

\noindent
\textbf{Projet: } Simulation de Chaleur sur un CPU\\
Projet c++ visant à créer une simulation du comportement d'une partie d'un CPU en utilisant des solutions numérique EDP.En implémentant deux modèles: stationnaires et dynamique.

\textbf{Emploie Etudiant} \\
Equipier polyvalent Mc Donald Erstein, Eté 2023.\\
livreur de journaux DNA,Eté 2022.

\end{document}
